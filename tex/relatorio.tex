\documentclass[conference]{IEEEtran}
\IEEEoverridecommandlockouts
% The preceding line is only needed to identify funding in the first footnote. If that is unneeded, please comment it out.
\usepackage{cite}
\usepackage{amsmath,amssymb,amsfonts}
\usepackage{algorithmic}
\usepackage{graphicx}
\usepackage{textcomp}
\usepackage{xcolor}
\usepackage[breaklinks]{hyperref}
\usepackage{microtype}

\def\BibTeX{{\rm B\kern-.05em{\sc i\kern-.025em b}\kern-.08em
    T\kern-.1667em\lower.7ex\hbox{E}\kern-.125emX}}
\begin{document}

\title{Implementação e Ataque da Cifra de Vigenère}

\author{\IEEEauthorblockN{1\textsuperscript{st} Leonardo Alves Riether}
\IEEEauthorblockA{\textit{Dep. Ciência da Computação} \\
\textit{Universidade de Brasilia}\\
Brasília, Brasil \\
190032413@aluno.unb.br}
}

\maketitle

\begin{abstract}
    \cite{teste}
This document is a model and instructions for \LaTeX.
This and the IEEEtran.cls file define the components of your paper [title, text, heads, etc.]. *CRITICAL: Do Not Use Symbols, Special Characters, Footnotes, 
or Math in Paper Title or Abstract.
\end{abstract}

\begin{IEEEkeywords}
component, formatting, style, styling, insert
\end{IEEEkeywords}

\section{Introdução}

\section{Estruturação do Projeto}
Neste trabalho, foi implementado um codificador e decodificador da cifra de
Vigenère em C++. O projeto foi compilado com GCC 12.1.0 e testado em Linux , mas
em princípio pode ser compilado em qualquer sistema com GCC que suporte C++17 e
executado tanto em sistemas baseados em Unix quanto no Windows. 

O projeto foi dividido em três partes:

\begin{enumerate}
    \item \textbf{main:} onde estão implementadas as funcionalidades de cifração
        e decifração da cifra de Vigenère, dada uma chave (ou arquivo de chave).
        Esse módulo é explicado na seção \ref{sec:implementation} e pode ser
        compilado com o comando \verb|make main|.
    \item \textbf{findkey:} onde está implementada a função de ataque da
        cifra de Vigenère, que encontra uma chave automaticamente, dada uma
        mensagem cifrada. Esse módulo é explicado na seção \ref{sec:attack} e
        pode ser compilado com o comando \verb|make findkey|. 
    \item \textbf{test:} possui alguns, poucos, testes, que podem ser executados
        com o comando \verb|make test|.
\end{enumerate}

\section{Implementação}
\label{sec:implementation}

\subsection{Operação de Mescla da Mensagem com a Chave}
Tradicionalmente, na cifra de Vigenère cada letra da mensagem é "mesclada" com
uma da chave, por meio da soma módulo 26 dos valores das letras (geralmente, o
valor de A é 0, o de B é 1, e assim por diante). Essa abordagem é interessante
quando se deseja cifrar texto, por sua simplicidade de implementação. No
entanto, existem algumas desvantagems:

\begin{itemize}
    \item Espaços e pontuação não são cifrados, portanto é fácil descobrir o tamanho de cada
        palavra.
    \item É mais difícil cifrar imagens, vídeos e arquivos binários que existem
        hoje em dia.
    \item Existem apenas 26 possibilidades para cada letra da mensagem
\end{itemize}

Dado isso, neste trabalho foi implementada uma variação da cifra de Vigenère
tradicional, que utiliza a operação de \textbf{xor bit-a-bit} entre cada byte da mensagem
e da chave. Com isso, é mais difícil descobrir o tamanho das palavras de um
texto cifrado, é possível cifrar qualquer tipo de arquivo, e existem 256
possibilidades para cada byte da chave, tornando essa cifra um pouco mais
difícil de quebrar.

\subsection{Codificação em Base 64}
\label{sec:base64}
Uma desvantagem de utilizar o xor é que o arquivo cifrado pode conter caracteres
ilegíveis. Para manter uma certa compatibilidade com a cifra de Vigenère
implementada com soma módulo 26, foi implementado um codificador e decodificador
de base 64. Os arquivos cifrados e codificados em base 64 utilizam apenas
caracteres legíveis, em troca de um aumento de aproximadamente 33\% do tamanho
do arquivo.

\subsection{Argumentos de Linha de Comando}
Para a execução do programa, após compilar com \verb|make main|, [TODO].

\section{Ataque}
\label{sec:attack}


\section{Conclusão}
\label{sec:conclusion}

% \begin{table}[htbp]
% \caption{Table Type Styles}
% \begin{center}
% \begin{tabular}{|c|c|c|c|}
% \hline
% \textbf{Table}&\multicolumn{3}{|c|}{\textbf{Table Column Head}} \\
% \cline{2-4} 
% \textbf{Head} & \textbf{\textit{Table column subhead}}& \textbf{\textit{Subhead}}& \textbf{\textit{Subhead}} \\
% \hline
% copy& More table copy$^{\mathrm{a}}$& &  \\
% \hline
% \multicolumn{4}{l}{$^{\mathrm{a}}$Sample of a Table footnote.}
% \end{tabular}
% \label{tab1}
% \end{center}
% \end{table}

% \bibliographystyle{IEEEtran}
\bibliographystyle{plain}
\bibliography{bib.bib}

\end{document}

