% Vimtex is great, but damn, running LaTeX locally is such a pain

\documentclass[conference]{IEEEtran}
% \IEEEoverridecommandlockouts
% The preceding line is only needed to identify funding in the first footnote. If that is unneeded, please comment it out.
% \usepackage{cite}
\usepackage[portuguese]{babel}
\usepackage{amsmath,amssymb,amsfonts}
\usepackage{algorithmic}
\usepackage{graphicx}
\usepackage{textcomp}
\usepackage{xcolor}
\usepackage[breaklinks]{hyperref}
\usepackage{microtype}
\usepackage{natbib}

\def\BibTeX{{\rm B\kern-.05em{\sc i\kern-.025em b}\kern-.08em
    T\kern-.1667em\lower.7ex\hbox{E}\kern-.125emX}}
\begin{document}

\title{Implementação e Ataque da Cifra de Vigenère}

\author{\IEEEauthorblockN{Leonardo Alves Riether}
\IEEEauthorblockA{\textit{Dep. Ciência da Computação} \\
\textit{Universidade de Brasilia}\\
Brasília, Brasil \\
190032413@aluno.unb.br}
}

\maketitle

\begin{abstract}
    \cite{englishfrequency}
This document is a model and instructions for \LaTeX.
This and the IEEEtran.cls file define the components of your paper [title, text, heads, etc.]. *CRITICAL: Do Not Use Symbols, Special Characters, Footnotes, 
or Math in Paper Title or Abstract.
\end{abstract}

\begin{IEEEkeywords}
    cifra, Vigenère, criptografia, xor
\end{IEEEkeywords}

\section{Introdução} % {{{

% }}}

\section{Estruturação do Projeto} % {{{
Neste trabalho, foi implementado um codificador e decodificador da cifra de
Vigenère em C++. O projeto foi compilado com GCC 12.1.0 e testado em Linux , mas
em princípio pode ser compilado em qualquer sistema com GCC que suporte C++17 e
executado tanto em sistemas baseados em Unix quanto no Windows. 

O projeto foi dividido em três partes:

\begin{enumerate}
    \item \textbf{main:} onde estão implementadas as funcionalidades de cifração
        e decifração da cifra de Vigenère, dada uma chave (ou arquivo de chave).
        Esse módulo é explicado na seção \ref{sec:implementation} e pode ser
        compilado com o comando \verb|make main|.
    \item \textbf{findkey:} onde está implementada a função de ataque da
        cifra de Vigenère, que encontra uma chave automaticamente, dada uma
        mensagem cifrada. Esse módulo é explicado na seção \ref{sec:attack} e
        pode ser compilado com o comando \verb|make findkey|. 
    \item \textbf{test:} possui alguns, poucos, testes, que podem ser executados
        com o comando \verb|make test|.
\end{enumerate}

Os comandos \verb|make| geram executáveis dentro do diretório \verb|bin|, por
exemplo \verb|bin/main| e \verb|bin/findkey|. Instruções mais detalhadas para a
execução do \verb|main| e \verb|findkey| podem ser encontradas nas seções
\ref{sec:exec-main} e \ref{sec:exec-attack}, respectivamente.  
% }}}

\section{Implementação} % {{{
\label{sec:implementation}

\subsection{Operação de Combinação da Mensagem com a Chave}
Tradicionalmente, na cifra de Vigenère cada letra da mensagem é "combinada" com
uma da chave, por meio da soma módulo 26 dos valores das letras (geralmente, o
valor de A é 0, o de B é 1, e assim por diante). Essa abordagem é interessante
quando se deseja cifrar texto, por sua simplicidade de implementação. No
entanto, existem algumas desvantagems:

\begin{itemize}
    \item Espaços e pontuação não são cifrados, portanto é fácil descobrir o tamanho de cada
        palavra.
    \item É mais difícil cifrar imagens, vídeos e arquivos binários que existem
        hoje em dia.
    \item Existem apenas 26 possibilidades para cada letra da mensagem
\end{itemize}

Dado isso, neste trabalho foi implementada uma variação da cifra de Vigenère
tradicional, que utiliza a operação de \textbf{xor bit-a-bit} entre cada byte da mensagem
e da chave. Com isso, é mais difícil descobrir o tamanho das palavras de um
texto cifrado, é possível cifrar qualquer tipo de arquivo, e existem 256
possibilidades para cada byte da chave, tornando essa cifra um pouco mais
difícil de quebrar.

\subsection{Codificação em Base 64}
\label{sec:base64}
Uma desvantagem de utilizar o xor é que o arquivo cifrado pode conter caracteres
ilegíveis. Para manter uma certa compatibilidade com a cifra de Vigenère
implementada com soma módulo 26, foi implementado um codificador e decodificador
de base 64. Os arquivos cifrados e codificados em base 64 utilizam apenas
caracteres legíveis, em troca de um aumento de aproximadamente 33\% do tamanho
do arquivo.

\subsection{Execução do Programa}
\label{sec:exec-main}
Após compilação do programa com \verb|make main|, é possível executá-lo com
alguns argumentos de linha de comando:

\begin{itemize}
    \item \textbf{-k} \textbf{--key}: Especifica uma string que será usada como
        chave da cifra.
    \item \textbf{-kf} \textbf{--key-file}: Espefifica um arquivo cujo conteúdo
        será usado como chave.
    \item \textbf{-i} \textbf{--input}: Determina qual o arquivo de mensagem (ou
        input). Se essa opção não for especificada, é utilizada a entrada padrão
        (stdin).
    \item \textbf{-o} \textbf{--output}: Determina qual o arquivo de cifra (ou
        output). Se essa opção não for especificada, é utilizada a saída padrão
        (stdout).
    \item \textbf{-i64} (\textit{resp.} \textbf{-o64}): Flag que, se usadas,
        indica que a entrada (\textit{resp.} saída) está codificada em base 64.
\end{itemize}

É necessário especificar uma chave, por meio de um argumento \textbf{-k} ou
\textbf{-kf}.

% }}}

\section{Ataque} % {{{
\label{sec:attack}

\subsection{Ideia}
É impossível projetar um ataque "universal" à cifra de Vigenère, visto que uma
mensagem aleatória, quando cifrada com uma chave qualquer, gera texto cifrado
aleatório. Esse caso é semelhante à aplicação de um one time
pad\cite{one-time-pad}, que é matematicamente "inquebrável", se usado corretamente.

Apesar disso, se a mensagem tiver alguma propriedade que conhecemos a priori, a
cifra de Vigenère pode ser quebrada. Assim, nessa seção consideraremos que a
mensagem cifrada é um texto relativamente longo, escrito em português ou inglês.
O ataque descrito também se aplica a outras línguas, mas focamos nessas duas.
Duas propriedades importantes de textos em português e inglês são:
\begin{enumerate}
    \item A grande maioria dos bytes é legível, diferentemente de arquivos de
        imagens ou executáveis.
    \item É esperado que as frequências dos caracteres sigam uma distribuição
        similar a outros textos escritos na mesma língua.
\end{enumerate}

Sabendo disso, podemos elaborar uma estratégia para encontrar uma chave
provável, dado um texto cifrado e a informação da língua em que a mensagem foi
escrita. A estratégia é dividida em três partes: encontrar o tamanho da chave,
encontrar a chave mais provável de um determinado tamanho, e por fim escolher a
chave com maior probabilidade de estar correta. Essas partes são explicadas nas
subseções \ref{sec:tamanhos}, \ref{sec:pontuacao}, \ref{sec:escolha}.

\subsection{Encontrando Tamanhos de Chave Mais Prováveis}
\label{sec:tamanhos}

Em textos grandes, é bem provável que certas algumas combinações de caracteres
se repitam frequentemente. Por exemplo, em inglês, o trigrama \textit{"the"}
provavelmente aparece muitas vezes. Nas vezes que um trigrama se repete na
mensagem e, por acaso, é cifrado com os mesmos índices da chave, o trigrama
cifrado também se repete!

Utilizamos esse fato para encontrar os tamanhos de chave mais prováveis com um
esquema de pontuação dos tamanhos. Primeiro, computamos a distância entre todas
as repetições de trigramas da cifra. Se uma repetição tem distância $\Delta $,
adicionamos um ponto a todos os divisores de $\Delta $. Por último, ordenamos os
inteiros com base em sua pontuação.

Um detalhe de implementação interessante é o algoritmo usado para encontrar os
divisores de um número. Nesse trabalho, foi utilizado o algoritmo de Pollard-Rho
\cite{pollard-rho}. A implementação foi retirada de \cite{pollard-rho-tiago}.

Esse método é chamado de Eliminação de Kasiski e é explicado em mais detalhes em
\cite{kasiski}.

\subsection{Função de Pontuação}
\label{sec:pontuacao}
Por meio da Eliminação de Kasiski, obtemos uma lista de tamanhos prováveis para
a chave. Agora, veremos como, dado um tamanho de chave $K$, obter a "melhor"
chave com tamanho $K$. Para determinar qual chave é "melhor" ou "pior",
definimos uma função de pontuação $P(C)$ que recebe uma chave e atribui a ela um
número. Quanto mais provável a chave, maior deve ser sua pontuação. 

Nesse ponto, notamos que, para uma chave de tamanho $K$, se dois índices $i, j$ da
mensagem são congruentes módulo $K$, eles foram cifrados com o mesmo byte da
chave. Sendo assim, podemos particionar os índices em classes de equivalência
módulo $K$ e pontuar cada classe separadamente.

Um modo de pontuar uma classe é comparar sua tabela de frequências com a tabela
esperada da língua. Como 

\begin{table}[htbp]
\caption{Frequência das Letras em Textos escritos em Inglês}
\begin{center}
\begin{tabular}{|c|c|c|c|}
\hline
    Letra & Frequência & Letra & Frequência \\
\hline
    E & 11.1607\% & M & 3.0129\% \\
    A & 8.4966\% & H & 3.0034\% \\
    R & 7.5809\% & G & 2.4705\% \\
    I & 7.5448\% & B & 2.0720\% \\
    O & 7.1635\% & F & 1.8121\% \\
    T & 6.9509\% & Y & 1.7779\% \\
    N & 6.6544\% & W & 1.2899\% \\
    S & 5.7351\% & K & 1.1016\% \\
    L & 5.4893\% & V & 1.0074\% \\
    C & 4.5388\% & X & 0.2902\% \\
    U & 3.6308\% & Z & 0.2722\% \\
    D & 3.3844\% & J & 0.1965\% \\
    P & 3.1671\% & Q & 0.1962\% \\
\hline
\end{tabular}
\label{tab:englishfrequency}
\end{center}
\end{table}

% De modo similar, um texto escrito em português tende a ter uma distribuição de
% frequências similar à da tabela apresentada em \ref{tab:portuguesefrequency},
% vista em \cite{wiki:portuguese-frequency}.

\begin{table}[htbp]
\caption{Frequência das Letras em Textos escritos em Português}
\begin{center}
\begin{tabular}{|c|c|c|c|}
\hline
    Letra & Frequência & Letra & Frequência \\
\hline
    A & 14.63\% & B & 1.04\% \\
    C & 3.88\% & D & 4.99\% \\
    E & 12.57\% & F & 1.02\% \\
    G & 1.30\% & H & 1.28\% \\
    I & 6.18\% & J & 0.40\% \\
    K & 0.02\% & L & 2.78\% \\
    M & 4.74\% & N & 5.05\% \\
    O & 10.73\% & P & 2.52\% \\
    Q & 1.20\% & R & 6.53\% \\
    S & 7.81\% & T & 4.34\% \\
    U & 4.63\% & V & 1.67\% \\
    W & 0.01\% & X & 0.21\% \\
    Y & 0.01\% & Z & 0.47\% \\
\hline
\end{tabular}
\label{tab:portuguesefrequency}
\end{center}
\end{table}

\subsection{Escolha da Chave Mais Provável}
\label{sec:escolha}

\subsection{Execução do Programa de Ataque}
\label{sec:exec-attack}

% }}}

\section{Conclusão} % {{{
\label{sec:conclusion}

% }}}

% \begin{table}[htbp]
% \caption{Table Type Styles}
% \begin{center}
% \begin{tabular}{|c|c|c|c|}
% \hline
% \textbf{Table}&\multicolumn{3}{|c|}{\textbf{Table Column Head}} \\
% \cline{2-4} 
% \textbf{Head} & \textbf{\textit{Table column subhead}}& \textbf{\teu conseextit{Subhead}}& \textbf{\textit{Subhead}} \\
% \hline
% copy& More table copy$^{\mathrm{a}}$& &  \\
% \hline
% \multicolumn{4}{l}{$^{\mathrm{a}}$Sample of a Table footnote.}
% \end{tabular}
% \label{tab1}
% \end{center}
% \end{table}

% \bibliographystyle{IEEEtran}
\bibliographystyle{unsrtnat}
\bibliography{bib}

\end{document}

